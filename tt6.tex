\documentclass[12pt, a4paper]{article}
\usepackage[T1,T2A]{fontenc}
\usepackage[utf8]{inputenc}
\usepackage[english,russian]{babel}
\usepackage{amsmath}
\usepackage{amsfonts}
\usepackage{amssymb}
\usepackage{makeidx}
\usepackage{tikz}
\usepackage{amsthm}
\usepackage{enumerate}
\usepackage[left=2cm,right=2cm,top=2cm,bottom=2cm,bindingoffset=0cm]{geometry}
\usepackage{proof}
\usetikzlibrary{graphs}
\usetikzlibrary{graphs.standard}
\usetikzlibrary{automata,positioning}
\begin{document}
\theoremstyle{definition}
\newtheorem{definition}{Определение}[section]
\newtheorem{note}{Замечание}[section]
\newtheorem*{example}{Пример}
\newtheorem{theorem}{Теорема}[section]
\newtheorem{axiom}{Аксиома}[section]
\newtheorem{lemma}[theorem]{Лемма}
\newtheorem{statement}{Утверждение}[section]
\newtheorem{oun_paragraph}{Пункт}[section]
\def\from#1{\par \parbox{0.7\textwidth}{\par \hfill\raggedleft \it #1}} 

	\begin{titlepage}
		\title{Лекция 6 \\ Реконструкция типов в просто типизированном лямбда-исчислении, комбинаторы}
		\date{}
	\end{titlepage}
		\maketitle
		
		
		
		
		\section{Лекция 6 \\ Реконструкция типов в просто типизированном лямбда-исчислении, комбинаторы}
		\subsection{Алгоритм вывода типов}
			Пусть есть: $?|-A:?$, хотим найти пару $\big \langle \text{контекст}, \text{тип} \big \rangle$\par
	\textbf{Алгоритм:}
	\begin{enumerate}
		\item Рекурсия по структуре формулы\par Построить по формуле $A$ пару $\big \langle E, \tau\big \rangle$, где\par $E-$набор уравнений, $\tau-$тип $A$
		\item Решение уравнения, получения подстановки $S$ и из решения $E$ и $S(\tau)$ получения ответа	
	\end{enumerate}
		Т.е. необохимо свести вывод типа к алгоритму унификации.\par
		\begin{oun_paragraph}Рассмотрим 3 случая\end{oun_paragraph}
			\begin{enumerate}
				\item $A\equiv x\implies\:\big \langle \{\}, \alpha_A\big\rangle$, где $\{\}-$пустой конекст, $\alpha_A-$новая переменная нигде не встречавшаяся до этого в формуле
				\item $A\equiv P\:Q\implies\big \langle E_P\cup E_Q\cup \{\tau_P=\tau_Q\rightarrow\alpha_A\}, \alpha_A\big \rangle$, где $\alpha_A-$новая переменная
				\item $A\equiv\lambda x.P\implies\big\langle E_P,\alpha_x\rightarrow\tau_P\big\rangle$
			\end{enumerate}
		\begin{oun_paragraph}Алгоритм унификации\end{oun_paragraph} 
			Рассмотрим $E-$набор уравнений, запишем все уравнения в алгебраическом виде т.е. $\alpha\rightarrow\beta\Leftrightarrow\rightarrow\alpha\beta$, затем применяем алгоритм унификации.
	\begin{lemma}
	Рассмотрим терм $M$ и пару $\big\langle E_M, \tau_M\big\rangle$, Если $\Gamma|-M:\rho$, то существует:
	\end{lemma}	
		\begin{enumerate}
		\item $S-$решение $E_M$ тогда $\Gamma=\{S(\alpha_x)\:|\:x\in FV(M)\}$, $FV-$множество свободных переменных в терме $M$, $\alpha_x-$ переменная полученная при разборе терма $M$\par
		$\rho=S(\tau_M)$
		\item Если $S-$ решение $E_M$, то $\Gamma|-M:\rho$,Доказательство$-$индукция по структуре терма $M$
	\end{enumerate}				
		 $\big \langle\Gamma,\rho\big\rangle-$основная пара для терма $M$, если 
		 \begin{enumerate}
			\item $\Gamma|-M:\tau$
			\item Если $\Gamma'|-M:\tau'$, то сущесвтует $S:\:S(\Gamma)\subset\Gamma'$
		 \end{enumerate}
		 \begin{example}
		\end{example}
		 	Рассмотрим терм: $\lambda\:f\lambda\:x.f(f(x))$, построим и пронумеруем его дерево разбора:\par
\begin{tikzpicture}
	\node [fill=none, draw=none] (A) at (9,0) {$\lambda\:f\lambda\:x.f(f(x))\enspace{\color{blue} (7)}$};
	\node [fill=none, draw=none] (B) at (7.5,-1) {$\lambda\:x.f(f(x))\enspace{\color{blue} (6)}$};
	\node [fill=none, draw=none] (C) at (11.5,-1) {$\lambda\:f$};
	\node [fill=none, draw=none] (D) at (5.5,-2) {$f(f(x))\enspace{\color{blue} (5)}$};
	\node [fill=none, draw=none] (E) at (9.5,-2) {$\lambda\:x$};
	\node [fill=none, draw=none] (F) at (3.5,-3) {$f(x)\enspace{\color{blue} (4)}$};
	\node [fill=none, draw=none] (G) at (7.5,-3) {$f\enspace{\color{blue} (3)}$};
	\node [fill=none, draw=none] (H) at (1.5,-4) {$f\enspace{\color{blue} (2)}$};
	\node [fill=none, draw=none] (I) at (4.5,-4) {$x\enspace{\color{blue} (1)}$};
	\path [->] (A) edge node[left] {} (B);
	\path [->] (A) edge node[left] {} (C);	
	\path [->] (B) edge node[left] {} (D);
	\path [->] (B) edge node[left] {} (E);
	\path [->] (D) edge node[left] {} (F);
	\path [->] (D) edge node[left] {} (G);
	\path [->] (F) edge node[left] {} (H);	
	\path [->] (F) edge node[left] {} (I);
\end{tikzpicture}\par
\begin{enumerate}
	\item $E_1=\big \langle \{\}, \alpha_x \big \rangle$
	\item $E_2=\big \langle \{\}, \alpha_f \big \rangle$
	\item $E_3=\big \langle \{\}, \alpha_f \big \rangle$
	\item $E_4=\big \langle \{\alpha_f=\alpha_x\rightarrow\alpha_1\}, \alpha_1 \big \rangle$
	\item $E_5=\big \langle \{\alpha_f=\alpha_x\rightarrow\alpha_1$\par$\alpha_f=\alpha_1\rightarrow\alpha_2\}, \alpha_2 \big \rangle$
	\item $E_6=\big \langle \{\alpha_f=\alpha_x\rightarrow\alpha_1$\par$\alpha_f=\alpha_1\rightarrow\alpha_2\}, \alpha_x\rightarrow\alpha_2 \big \rangle$
	\item $E_7=\big \langle \{\alpha_f=\alpha_x\rightarrow\alpha_1$\par$\alpha_f=\alpha_1\rightarrow\alpha_2\}, \alpha_f\rightarrow(\alpha_x\rightarrow\alpha_2) \big \rangle$
\end{enumerate}
	$E=\alpha_f=\alpha_x\rightarrow\alpha_1$\par$\alpha_f=\alpha_1\rightarrow\alpha_2$, решим полученную систему:\par 
	\begin{enumerate}
		\item приведем систему к алгебрачиескому виду и решим её:\par 
			\begin{enumerate}
			\item \[\begin{cases}
				\alpha_f=\rightarrow(\alpha_x\:\alpha_1)&\\
				\alpha_f=\rightarrow(\alpha_1\:\alpha_2)&\\
			\end{cases}\]
			\item \[
				\begin{cases}
				\rightarrow(\alpha_1\:\alpha_2)=\rightarrow(\alpha_x\:\alpha_1)
				\end{cases}\]
			\item \[
				\begin{cases}
				\alpha_1=\alpha_x&\\
				\alpha_2=\alpha_1
				\end{cases}\]
			\item \[
				\begin{cases}
				\alpha_1=\alpha_x&\\
				\alpha_2=\alpha_x
				\end{cases}\]
\end{enumerate}					
		\item  Получим \[S=\begin{cases}
						\alpha_f=\rightarrow(\alpha_x\:\alpha_1)&\\
						\alpha_1=\alpha_x&\\
						\alpha_2=\alpha_x&\\
				\end{cases}\]
		\item $\Gamma=\{\}$, так как в заданной формуле нет свободных переменных
		\item тип  терма $\lambda\:f\lambda\:x.f(f(x))$ является результат подстановки\par $S(\rightarrow\:\alpha_f\:(\alpha_x\rightarrow\alpha_2))$, получаем $\tau=(\alpha_x\rightarrow\alpha_x)\rightarrow(\alpha_x\rightarrow\alpha_x)$
	\end{enumerate}
	\begin{definition}Если существует последовательность редукций, приводящая терм $M$ в нормальную форму, то $M-$слабо нормализуем. (Т.е. при редуцировании терма $M$ мы можем не прийти в н.ф.)\end{definition} 
	\begin{definition}Если не существует бесконечной последовательности редукций терма $M$, то терм $M-$ сильно нормализуем.\end{definition} 
	 \begin{statement}
	 \end{statement}
	 \begin{enumerate}
	 	\item $KI\Omega-$ слабо нормализуема
	 	\item $\Omega-$ не нормализуема
	 	\item $II-$ сильно нормализуема
	 \end{enumerate}
	 \begin{lemma}Сильная нормализация влечет слабую.
	 \end{lemma}
	 \begin{statement}Любое $\lambda$ выражение можно записать с помощью комбинаторов $S$ и $K$, где\end{statement}
	 $S=\lambda\:x\lambda\:y\lambda\:z.(x\:z)(y\:z)$\par 
	 $K=\lambda\:x\lambda\:y.x$\par 
	\begin{statement}Соотношение комбинаторов с $\lambda$ исчислением:\end{statement}
	\begin{enumerate}
		\item $T(x)=x$
		\item $T(P\:Q)=T(P)T(Q)$
		\item $T(\lambda\:x.P)=K(T(P)),\enspace x\not\in FV(P)$
		\item $T(\lambda\:x.x)=I$
		\item $T(\lambda\:x\lambda\:y.P)=T(\lambda\:x.T(\lambda\:y.P))$
		\item $T(\lambda\:x.P\:Q)=S\:T(\lambda\:x.P)T(\lambda\:x.Q)$
	\end{enumerate}		
	 \begin{statement}Альтернативный базис:\end{statement}
	 \begin{enumerate}
		\item $B=\lambda\:x\lambda\:y\lambda\:z.x(y\:z)$
		\item $C=\lambda\:x\lambda\:y\lambda\:z.((x\:z)y)$
		\item $W=\lambda\:x\lambda\:y.((x\:y)y)$
	 \end{enumerate}

\end{document}