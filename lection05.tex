\section{Лекция 5 \\ Изоморфизм Карри-Ховарда (завершение), Унификация}
		\subsection{Изоморфизм Карри-Ховарда}
	\begin{definition}Изоморфизм Карри-Ховарда\end{definition}

	\begin{enumerate}
		\item $\Gamma\vdash M:\sigma$ влечет $|\Gamma|\vdash \sigma$ т.е. $|\{x_1:\Theta_1\:\hdots \:x_n:\Theta_n\}|=\{\Theta_1\:\hdots\:\Theta_n\}$
		\item Если $\Gamma\vdash\sigma$, то существует M и существует $\Delta$, такое что $|\Delta|=\Gamma$, что $\Delta \vdash M: \sigma$, где $\Delta=\{x_{\sigma} : \sigma\:|\:\sigma\in\Gamma  \}$
	\end{enumerate}
	\begin{example}
	$\{f\: :\:\alpha\:\rightarrow\:\beta,\:x\: :\:\beta\}\:\vdash\:f\:x\::\:\beta$ \\Применив, изоморфизм Карри-Ховарда получим: $\{\alpha\:\rightarrow\:\beta,\:\beta\}\:\vdash\:\beta$
	\end{example}
	\begin{proof}
		\par П.1 доказывается индукцией по длине выражения
			\begin{enumerate}
				\item $\Gamma,\:x:\Theta\vdash x:\Theta \qquad \Rightarrow_{KH} \qquad |\Gamma|,\:\Theta\vdash\:\Theta$
				\item $${\Gamma,\:x:\tau_1\vdash P:\tau_2 \over \Gamma \vdash \lambda x.\:P:\tau_1\:\rightarrow\:\tau_2} \qquad \Rightarrow_{KH} \qquad {|\Gamma|, \tau_1\vdash\tau_2\over |\Gamma|\:\vdash\:\tau_1\:\rightarrow\:\tau_2}$$
				\item $${\Gamma\vdash P:\tau_1\:\rightarrow\:\tau_2 \hspace{2em} \Gamma\vdash Q:\tau_1 \over \Gamma \vdash P\:Q:\tau_2} \qquad \Rightarrow_{KH} \qquad {|\Gamma|\vdash \tau_1\:\rightarrow\:\tau_2 \hspace{2em} |\Gamma|\vdash \tau_1 \over |\Gamma| \vdash \tau_2}$$
			\end{enumerate}
\par П.2 доказывается аналогичным способом но действия обратные.\\
Т.е. отношения между типами в системе типов могут рассматриваться как образ отношений между высказываниями в логической системе, и наоборот.
\end{proof}

\begin{definition}Расширенный полином:

	\[
    E(p,\:q)=
		\begin{cases}
    C,& \text{if }p=q=0\\
    p_1\:(p),& \text{if }q=0\\
    p_2\:(q),& \text{if }p=0\\
    p_3\:(p,\:q),& \text{if } p,\:q\neq0
		\end{cases}
	\]
	\[\text{ где }C-\text{константа, }\\p_1,\:p_2,\:p_3-\text{выражения, составленные из }*,\:+,\:p,\:q\text{ и констант.}\]
\end{definition}\par

	Пусть $\upsilon$ = $(\alpha\to\alpha)\to(\alpha\to\alpha)$, где $\alpha-$произвольный тип и пусть $F\in\Lambda\text{, что }F:\upsilon\to\upsilon\to\upsilon$, то существует расширенный полином $E$, такой что $\forall a,\:b\in \mathbb{N}$ $F(\overline{a},\:\overline{b})=_\beta\: \overline{E(a,\:b)}$, где $\overline{a}-$черчевский нумерал.
		\begin{theorem}
	У каждого терма в просто типизируемом $\lambda$ исчислении существует расширенный полином.
	\end{theorem}
		\begin{statement} Типы черчевских нумералов
	\begin{enumerate}
		\item $0:\lambda f\:\lambda x. \:x : a\:\rightarrow\:b\:\rightarrow\:b$
		\item $1:\lambda f\:\lambda x. \:f\:x : (a\:\rightarrow\:b)\:\rightarrow\:a\:\rightarrow\:b$
		\item $2:\lambda f\:\lambda x. \:f\:(f\:x) : (a\:\rightarrow\:a)\:\rightarrow\:a\:\rightarrow\:a$
		\item $\forall\:i,\:i\geq 2 \hspace{1em}\lambda f\:\lambda x. \:f(\:\hdots\:(f\:x)): (a\:\rightarrow\:a)\:\rightarrow\:a\:\rightarrow\:a$
	\end{enumerate}
	\end{statement}
	\begin{proof}
		Пункты $1, 2, 3-$ очевидно. Рассмотрим более подробно пункт 4:
		\par Разберем нумерал и рассмотрим два последних шага $-$


	\begin{table}[H]
			\begin{tabular}{ccccccc}
				&&&&\multicolumn{1}{c}{$f:a\:\rightarrow\:b  \vdash  x:a$}& $\{1\}$\\
				\cline{4-5}
				&&&\multicolumn{1}{c}{$f:a\:\rightarrow\:b  \vdash f\:x:b$}&&$\{2\}$\\
				\cline{3-4}
				&&\multicolumn{1}{c}{$f:a\:\rightarrow\:b  \vdash f\:(f\:x):\bot$}&&&$\{3\}$\\
				\cline{2-3}
				&\multicolumn{1}{c}{$\hdots$}\\
				\cline{1-2}
				\multicolumn{1}{c}{$\lambda f\:\lambda x. \:f(\:\hdots\:(f\:x))$}&&&&&&\\
			\end{tabular}
		\end{table}
		на шаге 3 становится понятно, что $f:a\:\rightarrow\:a$ и $x:a$
		\end{proof}


	\begin{statement}Основные задачи типизации $\lambda$ исчисления\end{statement}
		\begin{enumerate}
			\item \emph{Проверка типа$-$}выполняется ли $\Gamma\vdash M:\sigma$ для контекста $\Gamma\text{ терма }M\text{ и типа }\sigma$ (для проверки типа обычно откидывают $\sigma$ и рассматривают п.2).
			\item \emph{Реконструкция типа$-$}можно ли подставить вместо $?$ и $?_1$ в $?_1\vdash M:\:?$ подставить конкретный тип $\sigma$ в $?$ и контекст $\Gamma$ в $?_1$.
			\item \emph{Обитаемость типа$-$}пытается подобрать, такой терм $M$ и контекст $\Gamma$, что бы было выполнено $\Gamma\vdash M:\sigma$.
		\end{enumerate}
	\begin{definition}Алгебраический терм Выражение типа $$\Theta::=a\:|\:(f_k\:\Theta_1\:\dotsb\:\Theta_n)$$ где $a-$переменная, $(f_k\:\Theta_1\:\dotsb\:\Theta_n)-$применение функции \end{definition}
	\subsection{Уравнение в алгебраических термах $\Theta_1=\Theta_2$\\Система уравнений в алгебраических термах}
	\begin{definition}Система уравнений в алгебраических термах\end{definition}
	$
		\begin{cases}
			\Theta_1=\sigma_1&\\
			\vdots&\\
			\Theta_n=\sigma_n&\\
		\end{cases}
	$\par где $\Theta_i \text{ и } \sigma_i-\text{термы}$\par
	\begin{definition}$\{a_i\}=A-$множество перменных, $\{\Theta_i\}=T-$множество термов.\end{definition}
	\begin{definition}Подстановка$-$отображение вида: $S_0:A\to T$, которое является решением в алгебраических термах.\par $S_0(a)$ может быть либо $S_0(a)=\Theta_i\text{, либо }S_0(a)=a$.\end{definition}
	Доопределим $S$ на все $T$ т.е. $S:T\to T$, где \begin{enumerate}
		\item $S(a)=S_0(a)$
		\item $S(f(\Theta_1 \dotsb \Theta_k))=f(S(\Theta_1) \dotsb S(\Theta_k))$
	\end{enumerate}
	$S$ то же самое что и много $if'$ов либо $map$ строк.\par
	\begin{definition}Решить уравнение в алгебраических термах$-$найти такое $S$, что $S(\Theta_1)=S(\Theta_2)$\end{definition}
	\begin{example}\end{example}
		Заранее обозначим: $a,\:b-\text{переменные}\hspace{1em} f,\:g,\:h-\text{функции}$
		\begin{enumerate}
			\item $f\:(a\:(g\:b))=f\:(h\:e)\:d$ имеет решение $S(a)=h\:e\text{ и }S(d)=g\:b$
				\begin{enumerate}
					\item $S(f\:a\:(g\:b))=f\:(h\:e)\:(g\:b)$
					\item $S(f\:(h\:e)\:d)=f\:(h\:e)\:(g\:b)$
					\item $f\:(h\:e)\:(g\:b)=f\:(h\:e)\:(g\:b)$
				\end{enumerate}
			\item $f\:a=g\:b-$решений не имеет
		\end{enumerate}
		Таким образом, что бы существовало решение необходимо равенство строк полученной подстановки.\par
		\subsection{Алгоритм Унификации. Определения}
		\begin{enumerate}
			\item Система уравнений $E_1$ эквивалентна $E_2$, если они имеют одинаковые решения(унификаторы).
			\item Любая система $E$ эквивалентна некторому уравнению $\Sigma_1=\Sigma_2$.

	\begin{proof}
		Возьмем функциональный символ $f$, не использующийся в $E$, \par
		$
		E=\begin{cases}
			\Theta_1=\sigma_1&\\
			\vdots&\\
			\Theta_n=\sigma_n&\\
		\end{cases}
		$\par
		это же уравнение можно записать как$-$ $f\:\Theta_1\hdots\Theta_n=f\:\sigma_1 \hdots\sigma_n$\par
		Если существует подстановка $S$ такая, что\par $S(\Theta_i)=S(\sigma_i)\:\forall\:i$,
		то $S(f\:\Theta_1\hdots\Theta_n)=f\:S(\sigma_1)\hdots S(\sigma_n)$ \par Обратное аналогично.\end{proof}
		\item Рассмотрим операции
		\begin{enumerate}
			\item \textit{Редукция терма} \par
					Заменим уравнение вида$-f_1\;\Theta_1 \hdots\Theta_n=f_1\;\sigma_1\hdots\sigma_n$ на систему уравнений\par$\Theta_1=\sigma_1$\par$\vdots$\par$\Theta_n=\sigma_n$
			\item \textit{Устранение переменной} \par
			Пусть есть уравнение $x=\Theta$, заменим во всех остальных уравнениях переменную $x$ на терм $\Theta$.
		\end{enumerate}
		\begin{statement} Эти операции не изменяют множества решений.
		\end{statement}
		\begin{proof} Пункт $a\:-$ доказан выше, докажем теперь пункт $b:$\par Пусть есть решение вида $T = \begin{cases} a = \Theta_a &\\ \vdots \end{cases}$
		и уравнение вида $f\:a\:\hdots\:z\:=\:\Theta_c\:$, тогда, $T(f\:a\:\hdots\:z) = f\:T(a)\:\hdots\:T(z)$, которое в свою очередь является $f\: \Theta_a\:\hdots\:T(z)$
		\end{proof}
	\end{enumerate}
		\begin{definition}Система уравнений в разрешеной форме если \end{definition}
			\begin{enumerate}
				\item Все уравнения имеют вид $a_i=\Theta_i$
				\item Каждый из $a_i$ входит в систему уравнений только один раз
			\end{enumerate}
		\begin{definition}Система несовместима если\end{definition}
		\begin{enumerate}
			\item существует уравнение вида $f\:\Theta_1\hdots\Theta_n=g\:\sigma_1\hdots\sigma_n$, где $f\neq g$
			\item существует уравнение вида $a=f\:\Theta_1\hdots\Theta_n\:$, причем $a$ входит в какой-то из $\Theta_i$
		\end{enumerate}			

    \subsection{ Алгоритм унификации}
		\begin{enumerate}
		\item Пройдемся по системе, выберем такое уравнение, что оно удовлетворяет одному из условий:\begin{enumerate}
			\item Если $\Theta_i=a_i$, то перепишем, как $a_i=\Theta_i$, $\Theta_i-$не переменная
			\item $a_i=a_i-$ удалим
			\item $f\:\Theta_1\hdots\Theta_n=f\:\sigma_1\hdots\sigma_n-$  применим редукцию термов
			\item $a_i=\Theta_i-$Применим подстановку переменной $-$ подставим во все остальне уравнения $\Theta_i$ вместо $a_i$ (Если $a_i$ встречается в системе где-то еще)
		\end{enumerate}
		\item Проверим разрешима ли система, совместима ли система (два пункта несовместимости)
		\item повторим пункт 1
		\end{enumerate}

		\begin{statement} Алгоритм не изменяет множетва решений\end{statement}
		\begin{statement} Несовместная система не имеет решений\end{statement}
		\begin{statement} Если система имеет решение, то его разрешеная форма единствена\end{statement}
		\begin{statement} Система в разрешеной форме имеет решение:\end{statement}

	\begin{align*}
		\begin{cases}
			a_1=\Theta_1&\\
			\vdots&\\
			a_n=\Theta_n&\\
		\end{cases} & \text{имеет решение }-&\begin{cases}
			S_0(a_1)=\Theta_1&\\
			\vdots&\\
			S_0(a_n)=\Theta_n&\\
		\end{cases}\\
	\end{align*}
	\begin{statement}Алгоритм всегда закначивается\end{statement}
	\begin{proof}По индукции, выберем три числа $ \big \langle x\:y\:z\big \rangle$, где\par $x-$количество переменных, которые встречаются строго больше одного раза в левой части некоторого уравнения ($b$ не повлияет на $x$, а $a$ повлияет в уравнении $f(a(g\:a)\:b)=\Theta$),\par $y-$ количество функциональных символов в системе,\par $z-$количество уравнеий типа $a=a$ и $\Theta=b$, где $\Theta$ не переменная.\par Определим отношение $\leq$  между двумя кортежами,	 как $\big \langle x_1\:y_1\:z_1\big \rangle \leq \big \langle x_2\:y_2\:z_2\big \rangle$ если верно одно из следующих условий: \begin{enumerate}
	\item $x_1 < x_2$
	\item $x_1 = x_2 \:\& \:y_1 < y_2$
	\item $x_1 = x_2 \:\&  \:y_1 = y_2 \:\& \:z_1 < z_2$
	\end{enumerate}
		Заметим, что операции $(a)$ и $(b)$ всегда уменьшают $z$ и иногда уменьшают $x$.\\ Операция $(c)$ всегда уменьшает $y$ иногда $x$ и, возможно, увеличивает $z$. \\
		Операция $(d)$ всегда уменьшает $x$, и иногда увеличивает $y$. \\
		В сулчае если у сисетмы нет решений, алгоритм определит это на одном из шагов и завершится. \\
		Иначе с каждой операцией $a-d$ данная тройка будет уменьшаться, а так как $x,y,z\geq 0$, данный алгоритм завершится за конечное время.
		\end{proof} 

    \begin{example}\end{example}
			Исходная система
		\[
		E=
			\left \{
			\begin{tabular}{c}
				$g\:(x_2)=x_1$\\
				$f\:(x_1,\: h\: (x_1),\: x_2)=f\:(g\:(x_3),\: x_4,\: x_3)$
			\end{tabular}
			\right \}
		\]

			Применим пункт $(c)$ ко второму уравнению верхней системы получим:
			\[
		E=
			\left \{
			\begin{tabular}{c}
				$g\:(x_2)=x_1$\\
				$x_1=g\:(x_3)$\\
				$h\:(x_1)=x_4$\\
				$x_2=x_3$
			\end{tabular}
			\right \}
		\]

			Применим пункт $(d)$ ко второму уравнению верхней системы\\
				(оно изменит 1ое уравнение) получим:
			\[
			E=
			\left \{
			\begin{tabular}{c}
				$g\:(x_2)=g\:(x_3)$\\
				$x_1=g\:(x_3)$\\
				$h\:(g\:(x_3))=x_4$\\
				$x_2=x_3$

			\end{tabular}
			\right \}
		\]

			Применим пункт $(c)$ ко первому ур-ию\\
				и пункт $(a)$ к третьему уравнению верхней системы
			\[
			E=
			\left \{
			\begin{tabular}{c}
				$x_2=x_3$\\
				$x_1=g\:(x_3)$\\
				$x_4=h\:(g\:(x_3))$\\
				$x_2=x_3$
			\end{tabular}
			\right \}
		\]

			Применим пункт $(d)$ для первого уравнения к последнему уравнению, удалим последнее уравнение и \\
			получим систему в разрешеной форме

\[
			E=
			\left \{
			\begin{tabular}{c}
				$x_2=x_3$\\
				$x_1=g\:(x_3)$\\
				$x_4=h\:(g\:(x_3))$
			\end{tabular}
			\right \}
		\]

			Решение системы:
		\[		S=
			\left \{
			\begin{tabular}{c}
				$(x_1=g(x_3))$\\
				$(x_2=x_3)$\\
				$(x_4=h\:(g\:(x_3))))$\\
			\end{tabular}
			\right \}
		\]				
			
	\begin{statement} Если система имеет решение, алгоритм унификации приводит систему в разрешенную форму \end{statement}
	\begin{proof}
		От противного. \\ 
		Пусть алгоритм завершился и получившаяся система не в разрешенной форме.\\
		Тогда верно одно из следующих утверждений:
		\begin{enumerate}
			\item Одно из уравнений имеет вид отличный от $a_i = \Theta_i$, где $a_i$ -- переменная, то есть имеет следующий вид:
			\begin{enumerate}
				\item $f_i ~ \sigma_1 ... \sigma_n = f_i ~ \Theta_1 ... \Theta_n$ -- должна быть применена редукция термов $\Rightarrow$ алгоритм не завершился -- противоречие.
				\item $f_i ~ \sigma_1 ... \sigma_n = a_i$ -- должно быть применено правило разворота равентва -- противоречие.
			\end{enumerate}
			\item Все уравнения имеют вид $a_i = \Theta_i$, где $a_i$ -- переменная, но $a_i$ встречается в системе больше одного раза. \\
			В таком случае должно быть применено правило подстановки -- противоречие.
		\end{enumerate}
	\end{proof}
	\begin{definition} $S \circ T-$композиция подстановок, если $S \circ T=S\:(T\:(a))$\end{definition}
	\begin{definition} $S-$наиболее общий унификатор, если любое решение ($R$) системы $X$ может быть получено уточнением: $\exists\:T:R=T\circ S$\end{definition}
	
	\begin{statement} Алгоритм дает наиболее общий унификатор системы, если у нее есть решения. \end{statement}
	\begin{proof}
		Пусть S - решение полученное алгоритмом унификации \\
			  R - произвольное решение системы \\
			  $S_0$, $R_0$ - их сужения на множество переменных соответственно \\
			  \[E=\begin{cases}
			  ...&\\
			  a_i=\Theta_i&\\
			  ...\\
			  \end{cases}\]
			  где E - разрешенная форма исходной системы \\
		Согласно утверждению 6.9, алгоритм унификации приведет систему в разрешенную форму и полученное решение $S$ будет иметь сужение $S_0$, имеющее следующий вид:
		\begin{enumerate}
			\item $S_0(a_l) = \Theta_l$, если $a_l$ входит в левую часть $E$
			\item $S_0(a_r) = a_r$, если $a_r$ входит в правую часть $E$
		\end{enumerate}
		Рассмотрим какой вид может иметь $R$. Для этого достаотчно рассмотреть $R_0$. \\
		Заметим, что $R$ является решением $E$, так как $E$ эквивалентна исходной системе. \\
		Следовательно, $R_0$ имеет следующий вид:
		\begin{enumerate}
			\item $R_0(a_r) = \Theta$, где $\Theta$ -- произвольный терм, если $a_r$ входит в правую часть $E$
			\item $R_0(a_l) = \Theta_l[a_{r_1}:=R_0(a_{r_1}),...,a_{r_m}:=R_0(a_{r_m})]$, где $a_{r_k}$ -- переменная из правой части $E$, если $a_l$ входит в левую часть $E$
		\end{enumerate}
		Построим $T:R=T\circ S$. Зададим его через сужение $T_0$:
		\begin{enumerate}
			\item $T_0(a_r) = R_0(a_r)$, если $a_r$ входит в правую часть $E$
			\item $T_0 = id$, иначе
		\end{enumerate}
		Покажем, что $R=T\circ S$.
		Для этого достаточно доказать, что $R_0 = T \circ S_0$ \\
		Рассмотрим 2 случая:
		\begin{enumerate}
			\item $a_r$ - переменная из правой части $E$, \\
			тогда $(T \circ S_0)(a_r) = T(a_r) = T_0(a_r) = R_0(a_r)$
			\item $a_l$ - переменная из левой части $E$, \\
			тогда $(T \circ S_0)(a_l) = T(\Theta_l) = \Theta_l[a_{r_1}:=R_0(a_{r_1}),...,a_{r_m}:=R_0(a_{r_m})] = R_0(a_l)$
		\end{enumerate}
	Таким образом, мы для любого решения $R$ предъявили подстановку $T : R = T \circ S$, что является определением того, что $S$ - наиболее общий унификатор.
	\end{proof}
