\documentclass[12pt, a4paper]{article}
\usepackage{graphicx}
\usepackage{stmaryrd}
\usepackage[T1,T2A]{fontenc}
\usepackage[utf8]{inputenc}
\usepackage[english,russian]{babel}
\usepackage{amsmath}
\usepackage{amsfonts}
\usepackage{amssymb}
\usepackage{makeidx}
\usepackage{verbatim}
\usepackage{tikz}
\usepackage{amsthm}
\usepackage{enumerate}
\usepackage[left=2cm,right=2cm,top=2cm,bottom=2cm,bindingoffset=0cm]{geometry}
\usepackage{proof}
\usetikzlibrary{graphs}
\usetikzlibrary{graphs.standard}
\usetikzlibrary{automata,positioning}
\usepackage{float}
\title{Введение в Теорию Типов\\\it{Конспект лекций}}
\author{Штукенберг Д.~Г.\\Университет ИТМО}

\begin{document}

\theoremstyle{definition}
\newtheorem{definition}{Определение}[section]
\newtheorem{note}{Замечание}[section]
\newtheorem*{example}{Пример}
\newtheorem{theorem}{Теорема}[section]
\newtheorem{axiom}{Аксиома}[section]
\newtheorem{lemma}[theorem]{Лемма}
\newtheorem{statement}{Утверждение}[section]
\newtheorem{oun_paragraph}{Пункт}[section]
\def\from#1{\par \parbox{0.7\textwidth}{\par \hfill\raggedleft \it #1}} 


\newenvironment{epigraph}% 
{\begin{list}{}{\setlength{\leftmargin}{0.3\textwidth}}\item[]}% 
{\end{list}} 

\maketitle

\section{Введение}

Эти лекции были рассказаны студентам групп М3334--М3337, M3339
в 2018 году в Университете ИТМО, на Кафедре компьютерных технологий Факультета информационных технологий и программирования.

Конспект подготовили студенты Кафедры: Егор Галкин (лекции 1 и 2),
Илья Кокорин (лекции 3 и 4), Никита Дугинец (лекции 5 и 6), Степан Прудников (лекции 7 и 8).

(возможно, история сложнее)
	\section{Импликацонный фрагмент ИИП второго порядка}
 	\begin{center}
 		\begin{definition}
 			\large Назовем \textit{\underline{граммаикой ИИП второго порядка}} конструкцию вида: 
 		\end{definition}
 	\textbf{A} ::= 
 	(\textbf{A}) |
 	\textbf{p} |
 	\textbf{A} $\rightarrow$ \textbf{A} |
 	$\forall$\textbf{p}.\textbf{A} 
 	
 	\end{center}
 
 	\large В этой системе все остальные связки могут быть выражены через основные 4 представленные выше. Например $\perp$ представима в следующем виде
 	\begin{center}
 		{\textbf{\textsl\textit{\large $\forall$p.p}}} \\
 	\end{center}
 	
 	
 	 Также добавим два новых правила вывода дл\textsl{}я кваторов существования и два для всеобщности к уже существующим в ИИВ: \\ \\
 	
	Для квантора всеобщности: \\ 
 	
 	\Large{$\frac{\Gamma\vdash\phi}{\Gamma\vdash\forall p.\phi}$} \Large$(p\notin FV(\Gamma)) \qquad\qquad$ 
 	\Large{$\frac{\Gamma\vdash\forall p.\phi}{\Gamma\vdash\phi[p:=\Theta]}$}	\\
 	
 	 И две для квантора существования: \\
 	
 	\Large{$\dfrac{\Gamma\vdash\phi[p:= \psi]}{\Gamma\vdash\exists p.\phi}\qquad\qquad$} 
 	\Large{$\dfrac{\Gamma\vdash\exists p.\phi\qquad\Gamma, \phi\vdash\psi}{\Gamma\vdash\psi}$}\\ 
 	
 	
 	\begin{definition}
 		\large Грамматику ИИП второго порядка с переведенными выше правилами вывода назовем Импликационным фрагментом ИИВ второго порядка\\ 
 	\end{definition}
 	\large {С помощью этих правил вывода можно доказать что \textbf{${\perp = \forall p.p}$}
 		Действительно, воспользовавшись вторым правилом вывода квантора всеобщности для этого выражения мы можем вывести любое другое}
 	
 	\large С помощью правил вывода также можно доказать, что \\
 	$\phi\&\psi\equiv\forall a((\phi\rightarrow\psi\rightarrow a)\rightarrow a)$\\
 	$\phi\vee\psi\equiv\forall a((\phi\rightarrow a)\rightarrow(\psi\rightarrow a)\rightarrow a)$
 	
 	Докажем например, что
 	\begin{center}
 		 $\dfrac{\Gamma\vdash\perp}{\Gamma\vdash P}$
 	\end{center}
 	Воспользуемся вторым правилом вывода для квантора всеобщности
 	\begin{center}
		$\dfrac{\Gamma\vdash\forall\alpha.\alpha}{\Gamma\vdash\alpha[\alpha:=\phi]} $
 	\end{center}
 	
 	\section{Теория Моделей}
	Добавим к нашему исчислению модель. Напомню что модель это функция которая сопоставляет неокму терму элемент из множества истинностных значений. В нашем случае мы будем сопоставлять высказываниям элементы из множества $\llbracket\text{\textbf{И,Л}}\rrbracket$  по следующим правилам: \\

\begin{center}
	 	\large$\llbracket p\rrbracket=p$, т. е. $\llbracket p\rrbracket^{p = x} = x$ \\
\end{center}
 	
 	
 \begin{center}
 		\begin{equation*}
 		\llbracket p\rightarrow Q\rrbracket = 
 		\begin{cases}
 			\text{Л}, \llbracket p\rrbracket = \text{И}, Q = \text{Л} \\
 			\text{И}, \text{иначе}
 		\end{cases}
 	\end{equation*}
 \end{center}
 	
 	
 	\begin{equation*}
 		\llbracket\forall p.Q\rrbracket = 
 		\begin{cases}
 			\text{И}, \llbracket Q\rrbracket^{p=\text{л, и}} = \text{И} \\
 			\text{Л}, \text{иначе}
 		\end{cases}
 	\end{equation*}  
 	Эта модель корректна, но не полна.
 	
 	\section{Система F}
 	
\begin{definition}
	 	Под типом в системе F будем понимать следуюущее

 	
 	\begin{equation*}
 	\tau =
 	\begin{cases}
 	\alpha,\beta,\gamma ...\quad(\text{атомарные типы}) \\
 	\tau\rightarrow\tau \\
 	\forall\alpha.\tau\qquad(\alpha\text{ - переменная})
 	\end{cases}
 	\end{equation*}
 \end{definition}

 \begin{definition}
 		\large Введем определение грамматики в системе F:
 	\begin{center}
 		\large $\Lambda$ ::= x | $\lambda x^{\tau}.\Lambda$ | $\Lambda\Lambda$ | ($\Lambda$) | $\Lambda\alpha.\Lambda$ | $\Lambda\tau$ 
 	\end{center}
 \end{definition}
 	
 	где $\Lambda\alpha.\Lambda$ - типовая абстракция, явное указание того, что вместо каких-то типов мы можем подставить любые выражения, а $\Lambda\tau$ это применение типа. \\
 	
 	
 	Так пример типовой абстракции это: 
 	\begin{verbatim}
 		template<typename T>
 		class W {
 		    A x;
 		}
 	\end{verbatim}
 	
 	Типовая апликация это обьявление переменной класса с каким-то типом
 	
 	\begin{verbatim}
 	W<int> w_test;
 	\end{verbatim}\
 	
 	\begin{theorem}
 		Изоморфизм Карри - Хорварда :
    \begin{center}
 		$\Gamma\vdash_F M:\tau\Leftrightarrow |\Gamma|\vdash_{\forall, \rightarrow}\tau$ 
    \end{center}
 	
 	\end{theorem}
 	
 	
 	
 	В системе F определены следующие правила вывода: \\ \\
  	\noindent
 	\Large{$\dfrac{}{\Gamma,x:\tau\vdash x:\tau}\qquad\qquad$} 
 	\Large{$\dfrac{\Gamma\vdash M:\sigma\rightarrow\tau\qquad\Gamma\vdash N:\tau}{\Gamma\vdash M N:\tau}$}\\  \\ \\
 	\Large{$\dfrac{\Gamma,x:\tau\vdash M:\sigma}{\Gamma\vdash\lambda x^{\tau}.M:\tau\rightarrow\sigma}\quad(x\notin FV(\Gamma))$}\\ \\ \\
 	\Large{$\dfrac{\Gamma\vdash M:\sigma}{\Gamma\vdash\Lambda\alpha.M:\forall\alpha.\sigma}\quad(\alpha\notin FV(\Gamma))\qquad$}
 	\Large $\dfrac{\Gamma\vdash M:\forall\alpha.\sigma}{\Gamma\vdash M\tau:\sigma[\alpha:=\tau]}$
 	\\
 	
 	\large $\emph{Привидем пример}$. Покажем как выглядит в системе F левая проекция.
 	В простом типизированом $\lambda$ - исчислении $\pi_1$ имеет тип $\alpha\&\beta\rightarrow\alpha$. В системе F явно указывается, что элементы пары могут быть любыми и пишется сответственно $\forall\alpha.\forall\beta.\alpha\&\beta\rightarrow\alpha$. Само выражение для проекции также изменится и будет иметь вид  $\pi_1=\Lambda\alpha.\Lambda\beta.\lambda p^{\alpha\&\beta}.p\alpha\rm{T}$
 	
 	Давайте определим еще несколько понятий из простого $\lambda$-исчисления. \\
 	\emph{Начнем с $\beta$-редукции:}\\ \\
 	1. Типовая $\beta$-редукция: $(\Lambda\alpha.M^{\sigma})\tau\rightarrow_\beta M[\alpha:= \tau]:\sigma[\alpha:= \tau]$
 	2. Классическая $\beta$-редукция: $(\lambda x^{\sigma}.M)^{\sigma\rightarrow\tau}X\rightarrow_\beta M[x:=X]:\tau$ 
 	\\ \\
 	\emph{Выразим еще несколько функций} \\
 	
 	\noindent1. Не быввает М:$\perp$\\
 	2. Рассмотрим пару <P, Q> ::= $\Lambda\alpha.\lambda z^{\tau\rightarrow\sigma\rightarrow\alpha}.z P Q$ \\
 	Проекторы мы рассмотрели ранее. \\
 	3. $in_L(M^{\tau}) ::= \Lambda\alpha.\lambda u^{\tau\rightarrow\alpha}.\lambda\omega^{\sigma\rightarrow\alpha}.u M$\\
 	$ in_R(M^{\sigma}) ::= \Lambda\alpha.\lambda u^{\tau\rightarrow\alpha}.\lambda\omega^{\sigma\rightarrow\alpha}.u M$\\
 	

 	
 	(1) Чёрч Россер и прочетие теоремы доказуемые в строго-типизированом лямбда исчислении доказуемы и в системе F\\
 	(2) $\lambda_{(\forall, \rightarrow)}$ Система F силно нормализуема \\
 	(3) Y комбинатор не типизируем 
	(4) Исчисление неразрешимое, но не противоречивое
	\section{Ранг типа}
	\begin{definition}
		 Введем определение. Под {рангом типа} мы будем понимать число, получаемое по следующим правилам: \\
	    Rn(x) - множество всех типов x\\
	    Rn(0) - все типы без кванторов\\
	    Rn(x+1) = R(x) | R(x) $\rightarrow$ R(x+1) | $\forall\alpha.R(x+1)$
	\end{definition}
	 
	\textbf{Примеры}
	 1. $ \alpha\in $Rn(0) \\
	 2. $ \forall\alpha.\alpha \in$Rn(1)\\
	 3. $ (\forall\alpha.\alpha)\rightarrow(\forall\beta.\beta) \in$Rn(2), так как каждый тип вида $ \forall\alpha.\alpha \in$Rn(1) то по третьему правилу весь тип $ \in $Rn(2) \\

	\begin{definition}
		Тип с поверхностными кванторами - это любой тип вида $ \forall\alpha.\tau $ где в $ \tau $ отсутсвуют кванторы. Очевидно, что любой такой тип $ \in $Rn(1). Действительно, тип внутри кватнора точно имеет ранг 0. Навешивание одного или нескольких кванторов всеобщности увеличт его ранг на единицу.
	\end{definition}
	 
	 \section{Типовая схема}
	 Возьмем только типы с поверхностными кванторами(из Rn(1)). \\
	 Также можно формулу превратить любую формулу из Rn(1) в формулу с поверхностными кванторами. \\
	 Например:\\ 
	 $ \beta\rightarrow\forall\alpha.\alpha\equiv\forall\alpha.(\beta\rightarrow\alpha) $
	 \\
	 
	 \begin{definition}
	 	{типовой схемой} назовем выражение вида: 

\begin{center}
		 $ \sigma\equiv\forall\alpha_1.\forall\alpha_2.....\forall\alpha_n.t $ где t$ \in $Rn(0)
\end{center}
	 	 \end{definition}
 	 
	 Также будем считать что $ \sigma_1 <= \sigma_2 $\\
	 ($\sigma_2$ является спецификацией $\sigma_1$) если: \\\\
	 $\sigma_2\equiv\forall\alpha_1...\forall\alpha.\tau_1$\\
	 $\sigma_1\equiv\forall\beta_1...\beta_n.\tau_1[\alpha_1:=\Theta_1]...[\alpha_n:=\Theta_n]  $
	 
	 \noindent Например:\\
	 $ \forall\beta_1.\forall\beta_2.(\beta_1\rightarrow\beta_2)\rightarrow(\beta_1\rightarrow\beta_2) $\\
	 является спецификацией $ \forall\alpha.\alpha\rightarrow\alpha $\\
	 \section{Экзистенциальные типы}
	 1) $\dfrac{\Gamma\vdash\phi[\alpha:=\theta]}{\Gamma\vdash\exists\alpha.\phi}$\\ \\
	 2) $\dfrac{\Gamma\vdash\exists\alpha.\phi\qquad\Gamma,\phi\vdash\psi}{\Gamma\vdash\psi}$
	 
	  Экзистенциальные типы это способ инкапсуляции данных. Предположим что у нас есть стек с хранилищем типа $\alpha$ у которого определены следующие операции:\\\\
	 \textbf{empty}: $\alpha$\\
	 \textbf{push}: $\alpha\&\nu\rightarrow\alpha$\\
	 \textbf{pop}: $\alpha\rightarrow\alpha\&\nu$\\
	 
	 Тогда очевидно что тип stack$\equiv\alpha\&(\alpha\&\nu\rightarrow\alpha)\&(\alpha\rightarrow\alpha\&\nu)$
	 Но что если мы реализовали хранилище как-то по особенному, не меняя типов операций. Мы хотим скрыть данные о реализации, в частности о типе $\alpha$. Вместо деталей просто скажем что существует интерфейс, удовлетворяющий такому типу:\\$\exists\alpha.\alpha\&(\alpha\&\nu\rightarrow\alpha)\&(\alpha\rightarrow\alpha\&\nu)$
	 
	 \section{Абстрактные типы}	 
	 Предположим, что мы захотим создать стек в котором лежвт целые числа. Рассмотрим, как тогда будет выглядеть тип созданного стека: \\
	 \textbf{stack}$\equiv\forall\nu.\exists\alpha.\alpha\&(\alpha\&\nu\rightarrow\alpha)\&(\alpha\rightarrow\alpha\&\nu)$\\
 	По аналогии с правилом удаления квантора существования, можно определить правила вывода для выражений абстрактных типов: \\

	
 	$\dfrac{\Gamma \vdash M : \varphi[\alpha := \theta]}{\Gamma\vdash (\text{pack } M, \theta \text{ to } \exists \alpha . \varphi) : \exists \alpha.\varphi}\\ \\ \\
 	 $ Это правило вывода позволяет скрыть реализацию стека, так как если $\alpha$ это тип стека, то $\alpha[\nu := \theta]$ его конкретная реализация, например ArrayStack, LinkedListStack и подобные \\ \\
 	 $
 	\dfrac{\Gamma \vdash M : \exists \alpha . \varphi\qquad\Gamma, x : \varphi \vdash N : \psi}{\Gamma \vdash \text{abstype } \alpha \text{ with } x:\varphi \text{ in } M \text{ is } N:\psi}
	(\alpha \notin FV(\Gamma, \psi))$
	\\ \\
	Это правило вывода соответствует виртульному вызову стека какойто реализации, например: 
	\begin{verbatim}
		foo(Stack s) {
			...
		}
	\end{verbatim}
	Поскольку выводимые формулы выглядят слишком грамоздко, перепишем их, вспомнив, что: \\
	$\exists\alpha.\beta\equiv\forall\beta.(\forall\alpha.\sigma\rightarrow\beta)\rightarrow\beta$\\\\
	Тогда: \\
	$	\text{\textbf{pack} } M, \theta \text{ \textbf{to} } \exists \alpha . \varphi =
		\Lambda \beta . \lambda x^{\forall \alpha . \varphi \to \beta} . x \theta M \\
		\text{\textbf{abstype} } \alpha \text{ \textbf{with} } x:\varphi \text{ \textbf{in} } M \text{ \textbf{is} } N:\psi =
		M \psi (\Lambda \alpha . \lambda x ^ \varphi . N)
	$
	
	 \section{Типовая система Хиндли Милнера}
	   
	 Начнем с определения типа. \underline{Тип в системе Хиндли-Милнера}: \\ \\
	 Монотип - выражение в грамматике вида $\tau::=\alpha|\tau\rightarrow\tau|(\tau)$\\
	 Политип - выражение в грамматике вида $\sigma::=\tau|\forall\alpha.\sigma$\\
	 
	 \noindent Поэтому типы вида $\alpha\rightarrow\forall\beta.\beta$ - некорректны в системе ХМ\\
	 
	 \noindent Грамматика в системе Хиндли-Милнера имеет вид:\\
	 
	 \begin{center}
	 $\Lambda::=x | \lambda x.\Lambda | \Lambda\Lambda | (\Lambda) | \text{let x = }\Lambda \text{ in }\Lambda $ 
	 \end{center}
 		
 	 \noindent Обозначим контекст $\Gamma$ без типа x как $\Gamma_x$\\
 	 В новой системе получаем следуюущие правила вывода: \\ \\
 	 
 	 \noindent 1. Тавтология $\dfrac{}{\Gamma\vdash x.\phi}$ \\\\
 	
 	 \noindent 2. Уточнение $\dfrac{\Gamma\vdash e:\sigma}{\Gamma\vdash e:\sigma'}$ \\\\
 	 
 	 \noindent 3. Обобщение $\dfrac{\Gamma\vdash e:\sigma}{\Gamma\vdash e:\forall\alpha.\sigma}$ \\ \\
 	 \noindent 4. Абстракция $\dfrac{\Gamma_x, x:\tau'\vdash e:\tau}{\Gamma\vdash \lambda x.e:\tau'\rightarrow\tau}$ \\ \\
 	 \noindent 5. Применение $\dfrac{\Gamma\vdash e:\tau'\rightarrow\tau\qquad\Gamma\vdash e':\tau'}{\Gamma ee':\tau}$ \\ \\
 	 \noindent 6. Let $\dfrac{\Gamma\vdash e:\sigma\qquad\Gamma_x,x:\sigma\vdash e':\tau}{\Gamma\vdash\text{ \textbf{let} } x = e \textbf{ in }e':\tau}$ \\
 	 
 	 Хотя в системе Хиндли-Милнера (как и во всех рассматриваемых нами типовых системах) нельзя типизировать $\mathcal{Y} $-комбинатор,
 	 можно добавить его, расширив язык.
 	 Давайте определим его как $\mathcal{Y} f = f \left(\mathcal{Y} f\right)$.
 	 Какой у него должен быть тип? Пусть $\mathcal{Y}Y$ принимает $f$ типа $\alpha$, и возвращает нечто типа $\beta$,
 	 то есть $\mathcal{Y}: \alpha\to\beta$.
 	 Функция $f$ должна принимать то же, что возвращает $\mathcal{Y}$, так как результат $\mathcal{Y}$ передаётся в $f$,
 	 и возвращать она должна то же, что возвращает $\mathcal{Y}$, так как тип выражений с обоих сторон равенства должен быть одинаковый,
 	 то есть $f : \beta\to\beta$
 	 Кроме того, $\alpha$ и тип $f$ это одно и то же, $\alpha=\beta\to\beta$.
 	 После подстановки и заключения свободной переменной под квантор получаем $\mathcal{Y} : \forall\beta.(\beta\to\beta)\to\beta$.
 	 
 	 Через такой $\mathcal{Y}$ можно определять рекурсивные функции, и они будут типизироваться.
 	 
 
 	 



\end{document}
