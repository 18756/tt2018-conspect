\section{Лекция 3}

\subsection{Y-комбинатор}

\begin{definition}
	Комбинатором называется $\lambda$-выражение, не имеющее свободных переменных
\end{definition}

\begin{definition}($Y$-комбинатор)
	\[
	Y = \lambda f . (\lambda x . f (x x)) (\lambda x . f (x x))
	\]
\end{definition}

Очевидно, $Y$-комбинатор является комбинатором.

\begin{theorem}
	$Y f =_{\beta} f (Y f)$
	
	\begin{proof}
		$\beta$-редуцируем выражение $Y f$
		
		\begin{align*}
			 =_{\beta} \textcolor{magenta}{(\lambda f . (\lambda x . f (x x)) (\lambda x . f (x x)))}\textcolor{blue}{f} \\ =_{\beta} \textcolor{magenta}{(\lambda x . f (x x))} \textcolor{blue}{(\lambda x . f (x x))} \\ =_{\beta} f ((\lambda x . f (x x))(\lambda x . f (x x))) \\ =_{\beta} f(Y f)
		\end{align*}
		Так как при второй редукции мы получили, что $Y f =_{\beta} (\lambda x . f (x x))(\lambda x . f (x x))$
	\end{proof}	
\end{theorem}

Следствием этого утверждения является теорема о неподвижной точки для бестипового лямбда-ичисления

\begin{theorem}
	В лямбда-исчислении каждый терм $f$ имеет неподвижную точку, то есть такое $p$, что $f \; p =_{\beta} p$
	
	\begin{proof}
		Возьмём в качестве $p$ терм $Y f$. По предыдущей теореме, $f(Y f) =_{\beta} Y f$, то есть $Y f$ является неподвижной точкой для $f$. Для любого терма $f$ существует терм $Y f$, значит, у любого терма есть неподвижная точка.
	\end{proof}
\end{theorem}

\subsection{Рекурсия}