\section{Лекция 3}

\subsection{Y-комбинатор}

\begin{definition}
	Комбинатором называется $\lambda$-выражение, не имеющее свободных переменных
\end{definition}

\begin{definition}($Y$-комбинатор)
	\[
	Y = \lambda f . (\lambda x . f (x x)) (\lambda x . f (x x))
	\]
\end{definition}

Очевидно, $Y$-комбинатор является комбинатором.

\begin{theorem}
	$Y f =_{\beta} f (Y f)$
	
	\begin{proof}
		$\beta$-редуцируем выражение $Y f$
		
		\begin{align*}
			 =_{\beta} \textcolor{magenta}{(\lambda f . (\lambda x . f (x x)) (\lambda x . f (x x)))}\textcolor{blue}{f} \\ =_{\beta} \textcolor{magenta}{(\lambda x . f (x x))} \textcolor{blue}{(\lambda x . f (x x))} \\ =_{\beta} f ((\lambda x . f (x x))(\lambda x . f (x x))) \\ =_{\beta} f(Y f)
		\end{align*}
		Так как при второй редукции мы получили, что $Y f =_{\beta} (\lambda x . f (x x))(\lambda x . f (x x))$
	\end{proof}	
\end{theorem}

Следствием этого утверждения является теорема о неподвижной точки для бестипового лямбда-ичисления

\begin{theorem}
	В лямбда-исчислении каждый терм $f$ имеет неподвижную точку, то есть такое $p$, что $f \; p =_{\beta} p$
	
	\begin{proof}
		Возьмём в качестве $p$ терм $Y f$. По предыдущей теореме, $f(Y f) =_{\beta} Y f$, то есть $Y f$ является неподвижной точкой для $f$. Для любого терма $f$ существует терм $Y f$, значит, у любого терма есть неподвижная точка.
	\end{proof}
\end{theorem}

\subsection{Рекурсия}

С помощью $Y$-комбинатора можо определять рекурсивные функции, например, функцию, вычисляющую факториал Чёрчевского нумерала. Для этого определим вспомогательную функцию

$fact' = \lambda f. \lambda n. isZero\; n \; \overline{1} (mul \; n \; f((-1) n))$

Тогда $fact = Y fact'$

Для понимания того, как это работает, посчитаем $fact \; \overline{2}$

\begin{align*}
	fact \; \overline{2} \\ =_{\beta} Y \; fact' \; \overline{2}\\ =_{\beta} fact' (Y \; fact' \; \overline{2}) \\=_{\beta}\textcolor{magenta}{(\lambda f. \lambda n. isZero\; n \; \overline{1} (mul \; n \; f((-1) n)))} \textcolor{blue}{(Y \; fact') \overline{2}} \\
	=_{\beta}isZero\; \overline{2} \; \overline{1} (mul \; \overline{2} \; ((Y \; fact')((-1) \overline{2}))) \\ =_{\beta} mul \; \overline{2} \; ((Y \; fact')((-1) \overline{2})) \\ =_{\beta} mul \; \overline{2} \; (Y \; fact' \; \overline{1}) \\ =_{\beta} mul \; \overline{2} \; (fact' \;(Y \; fact' \; \overline{1}))
\end{align*}

Раскрывая $fact' \;(Y \; fact' \; \overline{1})$ так же, как мы раскрывали  $fact' \;(Y \; fact' \; \overline{2})$, получаем

\begin{align*}
	=_{\beta} mul \; \overline{2} \; (mul \; \overline{1} \; (Y \; fact' \; \overline{0}))
\end{align*}

Посчитаем $(Y \; fact' \; \overline{0})$.

\begin{align*}
	(Y \; fact' \; \overline{0}) \\ =_{\beta} fact' \; (Y \; fact') \; \overline{0} \\ =_{\beta} \textcolor{magenta}{(\lambda f. \lambda n. isZero\; n \; \overline{1} (mul \; n \; f((-1) n)))} \; \textcolor{blue}{(Y \; fact') \; \overline{0}} \\ =_{\beta}  isZero\; \overline{0} \; \overline{1} (mul \; \overline{0} \; ((Y \; fact'))((-1) \overline{0})) =_{\beta} \overline{1}
\end{align*}

Таким образом,

\begin{align*}
	fact \; \overline{2} \\ =_{\beta} mul \; \overline{2} \; (mul \; \overline{1} \; (Y \; fact' \; \overline{0})) \\=_{\beta} mul \; \overline{2} \; (mul \; \overline{1} \; \overline{1}) =_{\beta} mul \; \overline{2} \; \overline{1} =_{\beta} \overline{2}
\end{align*}