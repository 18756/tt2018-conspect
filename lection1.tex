\section{Лекция 1}

\subsection{$\lambda$-исчисление}

\begin{definition}[$\lambda$-выражение]
	$\lambda$-выражение "--- выражение, удовлетворяющее грамматике:
	\vspace{1mm}
	\begin{bnf}\begin{alignat*}{3}
		\Phi &::=& x \\
		| & \left(\Phi\right) \\
		| & \lambda{}x.\Phi \qquad && () \\
		| & \Phi \ \Phi         && () \\
	\end{alignat*}\end{bnf}
	\begin{enumerate}
		\item Аппликация левоассациативна.
		\item Абстракции жадные, едят все что могут.
		\vspace{1mm}
		\begin{example}
			$(\lambda{}x.(\lambda{}f.((f x) (f x) \lambda{}y.(y f))))$
		\end{example}
	\end{enumerate}
\end{definition}

\begin{definition}[$\alpha$-эквивалентность]
	$A=_{\alpha}B$, если имеет место одно из следующих условий:
	\begin{enumerate}
		\item $A\equiv{}x$, $B\equiv{}y$ (x,y "--- переменные) и $x\equiv{}y$
		\item $A\equiv{}P_{1}Q_{1}$, $B\equiv{}P_{2}Q_{2}$ и $P_{1}=_{\alpha}P_{2}$, $Q_{1}=_{\alpha}Q_{2}$
		\item $A\lambda{}x.P_{1}$, $B\lambda{}y.P_{2}$и $P_{1} [x\coloneqq{}t] =_{\alpha}P_2 [y\coloneqq{}t]$, где $t$ "--- новая переменная.
	\end{enumerate} 
\end{definition}

\begin{definition}[$\beta$-редекс]
	$\beta$-редекс "--- выражение вида: $\left(\lambda{}x.A\right)B$
\end{definition}

\begin{definition}[$\beta$-редукция]
	$A\to_{\beta}B$, если имеет мето одно из следующих условий:
	\begin{enumerate}
		\item $A\equiv{}P_{1}Q_{1}$, $B\equiv{}P_{2}Q_{2}$ и либо $P_{1}=_{\alpha}P_{2}$, $Q_{1}\to_{\beta}Q_{2}$, либо
		$P_{1}\to_{\beta}P_{2}$, $Q_{1}=_{\alpha}Q_{2}$
		\item $A\equiv\left(\lambda{}x.P\right) Q$, $B\equiv P[x\coloneqq{}Q]$ "--- Q свободна для подстановки вместо x в P 
	\end{enumerate}
	\begin{example} 
		$X\to_{\beta}X$, $\left(\lambda{}x.x\right) y\to_{\beta} y$
	\end{example}
	\begin{example}
		 $a \left(\lambda{}x.x\right) y\to_{\beta} a y$
	\end{example}
	\begin{example}
		$A\equiv\lambda{}x.P$, $B\equiv\lambda{}x.Q$, $P\to_{\beta}Q$
	\end{example}
\end{definition}

\subsection{Представление некоторых функций в лямбда исчислении}
Boolean значения легко представить в терминах $\lambda$-исчисления, к примеру
\begin{itemize}
	\item $True$   = $\lambda{}a\lambda{}b.a$ 
	\item $False$  = $\lambda{}a\lambda{}b.b$
\end{itemize}
Также мы можем выражать и более сложные функции \\
\newcommand{\If}{$\lambda{}c.\lambda{}t.\lambda{}e.(c t) e$}
\newcommand{\T}{$\lambda{}a\lambda{}b.a$}
\newcommand{\F}{$\lambda{}a\lambda{}b.b$}
	If = \If

\begin{example}

\end{example}

\subsection{Черчевские нумералы}

\begin{definition}[черчевский нумерал]
	\[
		\overline{n} = \lambda{}f.\lambda{}x.f^{n} x \text{, \quad где\quad}
		f^{n} x = 
		\begin{cases}
			f\left(f^(n-1) x\right) & \text{при } n > 0 \\
			x 						& \text{при } n = 0
		\end{cases}
	\]
\end{definition}




