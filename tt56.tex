\documentclass[12pt, a4paper]{article}
\usepackage[T1,T2A]{fontenc}
\usepackage[utf8]{inputenc}
\usepackage[english,russian]{babel}
\usepackage{amsmath}
\usepackage{amsfonts}
\usepackage{amssymb}
\usepackage{makeidx}
\begin{document}
	\begin{titlepage}
		\title{Лекция 5 \\ Изоморфизм Карри-Ховарда (завершение), Унификация}
		\date{}
	\end{titlepage}
		\maketitle
	\par \textbf{Определение}
	\\	
	Изоморфизм Карри-Ховарда
	\begin{enumerate}
		\item $\Gamma$ $\vdash$ M:$\sigma$ влечет |$\Gamma$|$\vdash$ $\sigma$
		\item $\Gamma$ $\vdash$ $\sigma$, то существует M и существует $\Delta$, такое что |$\Delta$|=$\Gamma$, что $\Delta$ $\vdash$M:$\sigma$, где $\Delta$=\{$x_{\sigma}$:$\sigma$|$\sigma$ $\in$ $\Gamma$  \}
	\end{enumerate}
	Рассмотрим пример:
	\{f:$\alpha\rightarrow\beta$, x:$\beta$\}$\vdash$fx:$\beta$ \\Применив изоморфизм Карри-Ховарда получим: \{$\alpha\rightarrow\beta$, $\beta$\}$\vdash\beta$\\
\par П.1 доказывается индукцией по длине выражения т.е. есть 3 правила вывода. убирая P и Q.
\par П.2 доказывается аналогичным способом но действия обратные.\\
Т.е. отношения между типами в системе типов могут рассматриваться как образ отношений между высказываниями в логической системе, и наоборот.
\\
\\
\par \textbf{Определение}
\par расширенный полином определяется формулой:
	\[
    E(p,q)= 
		\begin{cases}
    C,& \text{if }p=q=0\\
    p_1(p),& \text{if }q=0\\
    p_2(q),& \text{if }p=0\\
    p_3(p,q),& \text{if } p,q\neq0
		\end{cases}
	\]
	\[\text{, где }C-\text{константа, }\\p_1,p_2,p_3-\text{выражения, составленные из }*,+,p,q\text{ и констант}\]
	по сути расширенный полином это множество функций над натуральными числами(черчевскими нумералами).\par
	Пусть $\upsilon$ = $(\alpha\to\alpha)\to(\alpha\to\alpha)$, где $\alpha-$произвольный тип и пусть $F\in\Lambda\text{, что }F:\upsilon\to\upsilon\to\upsilon$, то существует расширенный полином $E$, такой что $\forall a,b\in \mathbb{N}$ $F(\overline{a},\overline{b})=_\beta \overline{E(a,b)}$, где $\overline{a}-$черчевский нумерал
	
	\par \textbf{Теорема}\par 
		У каждого терма в просто типизиреумом $\lambda$ исчислении существует расширенный полином.\par	
	\textbf{Основные задачи типизации $\lambda$ исчисления}\par
		\begin{enumerate}
			\item \emph{Проверка типа$-$}выполняется ли $\Gamma\vdash M:\sigma$ для контекста $\Gamma\text{ терма }M\text{ и типа }\sigma$ (для проверки типа обычно откидывают $\sigma$ и рассматривают п.2).
			\item \emph{Реконструкция типа$-$}можно ли подставить вместо $?$ и $?_1$ в $?_1\vdash M:?$ подставить конкретный тип $\sigma$ в $?$ и контекст $\Gamma$ в $?_1$.
			\item \emph{Обитаемость типа$-$}пытается подобрать, такой \textbf{замкнутый} терм $M$ и контекст $\Gamma$, что бы было выполнено $\Gamma\vdash M:\sigma$.
		\end{enumerate}			
	\par Определение \textbf{Алгебраический терм}$-$выражение типа $\Theta=a|(f_k \Theta_1\dotsb\Theta_n)$, \par где $а-$переменная, $(f_k \Theta_1\dotsb\Theta_n)-$применение функции\par
	Примеры:
	\begin{enumerate}
	\item $(f a b ( g a))$ 
	\item Известно, что $\to-$функция, тогда выражение $((a\to b)\to c)$ $\Longleftrightarrow$ $(\to (\to a b) c)$
\end{enumerate}		
	\par \textbf{Уравнение в алгебраических термах} $\Theta_1=\Theta_2$\par
	\par \textbf{Система уравнений в алгебраических термах}\par
	$		
		\begin{cases}			
			\Theta_1=\sigma_1&\\
			\vdots&\\
			\Theta_n=\sigma_n&\\
		\end{cases}
	$\par где $\Theta_i \text{ и } \sigma_i-\text{термы}$\par
	\par Определение $\{a_i\}=A-$множество перменных, $\{\Theta_i\}=T-$множество термов.\par
	\par Определение \textbf{Подстановка}$-$отображение вида: $S_0:A\to T$, которое является решением в алгебраических термах.\par Т.е. $S_0-$конечное множество переменных $a_1\dotsb a_n$ на которых $S_0(a_i)=\Theta_i\text{ либо }S_0(a_i)=a_i$.\par
	Доопределим $S$ на все $T$ т.е. $S:T\to T$, где \begin{enumerate}
		\item $S(a)=S_0(a)$
		\item $S(f(\Theta_1 \dotsb \Theta_k))=f(S(\Theta_1) \dotsb S(\Theta_k))$
	\end{enumerate}
	По сути $S$ тоже самое что и много $if'$ов либо $map$ строк\par 
	Определение \textbf{Решить уравнение в алгебраических термах}$-$найти такое $S$, что $S(\Theta_1)=S(\Theta_2)$\par 
	Например:\par 
		Заранее обозначим: $a,b-\text{переменные},$ $f,g,h-\text{функции}$
		\begin{enumerate}
			\item $f(a(gb))=f(he)d$ имеет решение $S(a)=he\text{ и }S(d)=gb$
				\begin{enumerate}
					\item $S(fa(gb))=f(he)(gb)$
					\item $S(f(he)d)=f(he)(gb)$
					\item $f(he)(gb)=f(he)(gb)$
				\end{enumerate}
			\item $fa=gb-$решений не имеет
		\end{enumerate}
		Таким образом, что бы существовало решение необходимо равенство строк полученной подстановки\par
		\textbf{Алгоритм Унификации}\par
		
\par \textbf{Ссылки}
\begin{enumerate}
\item https://www.quora.com/What-is-an-intuitive-explanation-of-the-Curry-Howard-correspondence
\item https://habr.com/post/269907/
\item https://arxiv.org/pdf/cs/0701022.pdf
\end{enumerate}

\end{document}