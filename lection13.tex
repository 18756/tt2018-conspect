\section{Лекция 13}

\subsection{Теорема Диаконеску}
\begin{theorem}[Диаконеску]
    Аксиома выбора влечёт закон исключённого третьего.
\end{theorem}

\begin{proof}

Для любого утверждения $P$ по аксиоме выделения мы можем построить два множества:
\[
    A = \set{x \in \set{0, 1} \mid \left(x = 0\right) \lor P} \qquad
    B = \set{x \in \set{0, 1} \mid \left(x = 1\right) \lor P}
\]
По аксиоме выбора мы знаем, что их декартово произведение непусто.
Иначе говоря, существует функция $f : \set{A, B} \to \set{0, 1}$, что
\[
    f(A) \in A \& f(B) \in B
\]
Это, по определению двух множеств, эквивалентно
\[
    \left(f(A) = 0 \vee P\right) \& \left(f(B) = 1 \vee P\right)
\]
Из этого следует, что
\begin{gather}
    \left(f(A) \neq f(B)\right) \vee P \label{firstOption} \tag{$*$}
\end{gather}
Однако, по принципу объёмности $P \to \left(A=B\right)$. Значит, $P \to \left(f(A) = f(B)\right)$. Значит,
\begin{gather}
    \left(f(A) \neq f(B)\right) \to \neg P \label{secondOption} \tag{$**$}
\end{gather}
Из \ref{firstOption} и \ref{secondOption} можно вывести $P \vee \neg P$.

\end{proof}

\subsection{Сетоид}
\begin{definition}[Сетоид]
    \[
        \left<C : \mathrm{Type}, \mathrm{Eq} : C \to C \to \mathrm{Type}, P : \mathrm{IsEquivalence}~\mathrm{Eq}\right>
    \]
    $Eq$ "--- отношение эквивалентности, $P$ "--- доказательство этого факта.
\end{definition}
\begin{definition}[Экстенциональность]
    \[
        f : \left< A, =_A, P_A\right> \to \left<B, =_B, P_B\right>
    \]
    $f$ экстенциональна ($\mathrm{Ext}~f$), если из $x =_A y$ следует $f x =_B f y$
\end{definition}

Пример сетоида "--- целые числа:
\[
    \left< \left<p : \mathrm{Nat}, n : \mathrm{Nat}\right>, \overline{Eq(a+d=b+c)}, \mathrm{IsEquivalence} Eq\right>
\]


Интенциональное исчисление "--- это исчисление, равенство в котором является структурным.

Экстенциональное исчисление "--- это исчисление, равенство в котором задаётся отношением эквивалентности.