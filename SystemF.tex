\documentclass[10pt,a4paper]{article}
\usepackage[utf8]{inputenc}
\usepackage{amsmath}
\usepackage[russian]{babel}
\usepackage{amsfonts}
\usepackage{amssymb}
\usepackage{graphicx}
\usepackage{stmaryrd}
\title{Интуиционистское исчисление второго порядка. Система F}

\begin{document}
	\maketitle
	\section{Импликацонный фрагмент ИИП второго порядка}
	\large Назовем \textit{\underline{граммаикой ИИП второго порядка}} конструкцию вида: \\ \\
	\textbf{A} ::= 
	 (\textbf{A}) |
	 \textbf{p} |
	 \textbf{A} $\rightarrow$ \textbf{A} |
	 $\forall$\textbf{p}.\textbf{A} | 
	 $\exists$\textbf{p}.\textbf{A} |
	 $\perp$ |
	 \textbf{A}$\land$\textbf{A} |
	 \textbf{A}$\lor$\textbf{A} \\
	
	\large Стоит отетить, что последние четыре свзки нам не очень важны, так как могут быть выражены через первые четыре.Например $\perp$ это ни что иное как: 
	\begin{center}
			{\textbf{\textsl\textit{\large $\forall$p.p}}} \\
	\end{center}

   
   	\large Также добавим два новых правила вывода дл\textsl{}я кваторов существования и два для всеобщности к уже существующим в ИИВ: \\ \\
  	
  	\emph{Для квантора всеобщности:} \\ 
  	
  	\huge{$\frac{\Gamma\vdash\phi}{\Gamma\vdash\forall p.\phi}$} \Large$(p\notin FV(\Gamma)) \qquad\qquad$ 
   	\huge{$\frac{\Gamma\vdash\forall p.\phi}{\Gamma\vdash\phi[p:=\Theta]}$}	\\
   	
	\emph{\large И две для квантора существования: \\}
 	
 	\Large{$\dfrac{\Gamma\vdash\phi[p:= \psi]}{\Gamma\vdash\exists p.\phi}\qquad\qquad$} 
	\Large{$\dfrac{\Gamma\vdash\exists p.\phi\qquad\Gamma, \phi\vdash\psi}{\Gamma\vdash\psi}$}\\ 
   	
   	
   	\large Грамматику ИИП второго порядка с переведенными выше правилами вывода назовем \underline{Импликационным фрагментом ИИВ второго порядка}  \\ 
   	\large {С помощью этих правил вывода можно доказать что \textbf{${\perp = \forall p.p}$}
   	Действительно, воспользовавшись вторым правилом вывода квантора всеобщности для этого выражения мы можем вывести любое другое}
   
   	\large С помощью правил вывода также можно доказать, что \\
   	$\phi\&\psi\equiv\forall a((\phi\rightarrow\psi\rightarrow a)\rightarrow a)$\\
   	$\phi\vee\psi\equiv\forall a((\phi\rightarrow a)\rightarrow(\psi\rightarrow a)\rightarrow a)$
   
   	\section{Теория Моделей}
   	\textsl{}
   	\large$\forall$p.p$\rightarrow$p\\
   	\large$\textcircled{1} \\
   	  \llbracket p\rrbracket=p$, т. е. $\llbracket p\rrbracket^{p = x} = x$ \\


   	\large\noindent $\textcircled{2}$
   	\begin{equation*}
   		\llbracket p\rightarrow Q\rrbracket = 
   		\begin{cases}
   			\text{Л}, \llbracket p\rrbracket = \text{И}, Q = \text{Л} \\
   			\text{И}, \text{иначе}
   		\end{cases}
   	\end{equation*}
   	
   	\large \noindent $\textcircled{3}$
 	   	\begin{equation*}
 			\llbracket\forall p.Q\rrbracket = 
 			\begin{cases}
 				\text{И}, \llbracket Q\rrbracket^{p=\text{л, и}} = \text{И} \\
 				\text{Л}, \text{иначе}
 			\end{cases}
 		\end{equation*}  
 	Эта модель корректна, но не полна.
 	
	\section{Система F}
	\large Введем определение грамматики в системе F:
	\begin{center}
		\large $\Lambda$ ::= x | $\lambda x^{\tau}.\Lambda$ | $\Lambda\Lambda$ | ($\Lambda$) | $\Lambda\alpha.\Lambda$ | $\Lambda\tau$ 
	\end{center}
	
	где $\Lambda\alpha.\Lambda$ - типовая абстракция, явное указание того, что вместо каких-то типов мы можем подставить любые выражения, а $\Lambda\tau$ это применение типа.
	Собственно под типом в системе F подразумевается следуюущее:
	
	\begin{equation*}
		\tau =
		\begin{cases}
			\alpha,\beta,\gamma ...\quad(\text{атомарные типы}) \\
			\tau\rightarrow\tau \\
			\forall\alpha.\tau\qquad(\alpha\text{ - переменная})
		\end{cases}
	\end{equation*}
	
	В системе F определены следующие правила вывода: \\ 
	
	\huge{$\dfrac{}{\Gamma,x:\tau\vdash x:\tau}\qquad\qquad$} 
	\Large{$\dfrac{\Gamma\vdash M:\sigma\rightarrow\tau\qquad\Gamma\vdash N:\tau}{\Gamma\vdash M N:\tau}$}\\  \\
	\Large{$\dfrac{\Gamma,x:\tau\vdash M:\sigma}{\Gamma\vdash\lambda x^{\tau}.M:\tau\rightarrow\sigma}\quad(x\notin FV(\Gamma))$}\\ \\
	\Large{$\dfrac{\Gamma\vdash M:\sigma}{\Gamma\vdash\Lambda\alpha.M:\forall\alpha.\sigma}\quad(\alpha\notin FV(\Gamma))\qquad$}
	\Large $\dfrac{\Gamma\vdash M:\forall\alpha.\sigma}{\Gamma\vdash M\tau:\sigma[\alpha:=\tau]}$
	\\
	
	\large $\emph{Привидем пример}$. Покажем как выглядит в системе F левая проекция.
	В простом типизированом $\lambda$ - исчислении $\pi_1$ имеет тип $\alpha\&\beta\rightarrow\alpha$. В системе F явно указывается, что элементы пары могут быть любыми и пишется сответственно $\forall\alpha.\forall\beta.\alpha\&\beta\rightarrow\alpha$. Само выражение для проекции также изменится и будет иметь вид  $\pi_1=\Lambda\alpha.\Lambda\beta.\lambda p^{\alpha\&\beta}.p\alpha\rm{T}$
	
	Давайте определим еще несколько понятий из простого $\lambda$-исчисления. \\
	\emph{Начнем с $\beta$-редукции:}\\ \\
	1. Типовая $\beta$-редукция: $(\Lambda\alpha.M^{\sigma})\tau\rightarrow_\beta M[\alpha:= \tau]:\sigma[\alpha:= \tau]$
	2. Классическая $\beta$-редукция: $(\lambda x^{\sigma}.M)^{\sigma\rightarrow\tau}X\rightarrow_\beta M[x:=X]:\tau$ 
	\\ \\
	\emph{Выразим еще несколько функций} \\
	
	\noindent1. Не быввает М:$\perp$\\
	2. Рассмотрим пару <P, Q> ::= $\Lambda\alpha.\lambda z^{\tau\rightarrow\sigma\rightarrow\alpha}.z P Q$ \\
	Проекторы мы рассмотрели ранее. \\
	3. $in_L(M^{\tau}) ::= \Lambda\alpha.\lambda u^{\tau\rightarrow\alpha}.\lambda\omega^{\sigma\rightarrow\alpha}.u M$\\
	$ in_R(M^{\sigma}) ::= \Lambda\alpha.\lambda u^{\tau\rightarrow\alpha}.\lambda\omega^{\sigma\rightarrow\alpha}.u M$\\
	
	В системе F задачи о реконструкции, проверке и общезначности типа неразрешимы\\
	
	Основные теоремы к доказательству: \\
	(1) Чёрч Россер\\
	(2) $\lambda_{(\forall, \rightarrow)}$ силно нормализуема \\
	(3) Y комбинатор не типизируем
   \end{document}
